%\section{Work Process}\label{sec:work-process}

When designing the solution, the first step is to determine the general architecture. In the process of determining the architecture, various tasks in the system first have to be described. From that point, the communication between the devices will have to be determined, and the protocols for this communication will be described.

After making decisions about the architecture, tasks, and protocols, the next step is to research which hardware platform will be the most ideal for this project. This will open the discussion for what kind of \sdevs~that can be used when testing the system. At last a user friendly UI will be designed.

Seeing as the project will pass through multiple iterations, the initial implementation will start after the first major design decisions have been taken. This will ensure a smooth transition into the implementation chapter, where the implemented features will be explained in further details. These description will include every implemented feature, from front end to back end.

Throughout this shift from design to implementation, the group will be split into. One part will focus on the initial stages of the implementation such as setting up the environment and starting on some basis tasks. The second part of the group will continue on the design chapter and slowly start on the implementation as well. These groups will change approximately one person daily, resulting in everybody getting time on both parts of the project. These groups will continue until the end of the implementation chapter. At this point more people will shift to the coding group, while the remaining people will read through and correct the current iterations of the design and implementation chapters.

The more advanced programming tasks will be done with pair programming. Having a designated discussion partner when programming have proven very helpful in earlier projects. This will of course only be done when working with more advanced methods and algorithms.

When the system reaches a stable point, the testing phase will begin. Here the system will be tested on its abilities to fulfill the requirements set in the problem statement, see \secref{sec:problem-statement}. All of the results will be written down, and the most important ones will be typed up for the report.

