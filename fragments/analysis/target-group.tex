\section{Target Group} \label{sec:target-group}

The existing solutions are primarily targeting upper middle class citizens, as the components and devices needed are quite expensive. Taking into consideration that the technology is rather new and constantly evolving, the pricing is understandable. However there might be people with less money to spend or people who already own different \sdevs, who are also interested in having an automated home.

Another possible target group are handicapped and chronically ill people~\citep{overview-smart-home-environment}. A Smart Home system could drastically improve their quality of life, by giving these people easier access to devices. Some problems they could experience include, reaching switches, bending down to reach devices at floor level, and blind people that cannot change channels etc~\citep{disabled-smart-home}. This could also affect temporary ill people, that for a period of time might be constrained to their beds or the like. If these people could access their house devices through a Smart Home solution, a lot of trouble could be avoided. 

The problem is that a lot of the Smart Home systems only work with their own products and an expensive \hub~is required for the Smart Home system to even work. As such, people may not be able to make such an investment or they may not be able to connect all of their current devices into one system. Overcoming these problems by avoiding a \hub~and increasing the interoperability could make home automation more available and appealing to a broader audience than it currently is. 

