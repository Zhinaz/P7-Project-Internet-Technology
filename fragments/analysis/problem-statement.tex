\section{Problem Statement} \label{sec:problem-statement}

For each of the four use cases from \secref{sec:use-cases} the main drawbacks are: 
\begin{itemize}
\item John's usability problem with using multiple \phones~to control his home.
\item Arthur's problem with a defective hub that the entire Smart Home relies on.
\item Sarah's problem with unstable Internet and dependency on Internet access, rather than just Wi-Fi connection to the devices.
\item Alice's security problem with the neighbour abusing her \sdevs.
\end{itemize}

As found in the existing solutions in \secref{sec:existing-solutions}, there exists no solution which controls various \sdevs~without utilising a \hub.
Furthermore, only a few solutions have implemented authentication, none of which implement user access levels in their solution.

Based on the initiating problem, the use cases, and existing solutions, the focus of this project is:
\begin{changemargin}{1cm}{1cm} 
To what extend could one make a Smart Home which satisfies the following requirements:
\begin{itemize}
\item Interact with hardware from different manufactures.
\item Provide a common \phone~for the users.
\item Run on a single device without external dependencies, without the need for a central hub. 
\item User authentication and different access levels for users in the system.
\end{itemize}
\end{changemargin} 




%\begin{table}[H]
%	\centering
%	\begin{narrow}{1cm}{1cm}
%
%	\rule{\linewidth}{0.035cm}
%	\begin{center}
%	\textbf{Problem Statement} 
%	\end{center}
%	\rule{\linewidth}{0.035cm}
%    
%	\medskip\noindent \textit{To what extend could one make a Smart Home which satisfies the following requirements:
%\begin{itemize}
%\item Interact with hardware from different suppliers.
%\item Supply a common interface for the users.
%\item Run on a single device without external dependencies, without the need for a central hub. 
%\item User authentication and different access levels for users in the system.
%\end{itemize}}
%
%	\rule{\linewidth}{0.035cm}      
%	\end{narrow}
%\end{table}


%How could one make an adaptable smart home that can:
%\begin{itemize}
%\item handle various devices from different suppliers, combined in an accessible user control interface
%\item be accessed by a single device, without the need from a central hub
%\end{itemize}

%Based on the initiating problem, the use cases, and the research within the field of existing Smart Home technologies, this project focuses on the following problems:

%\begin{changemargin}{0.5cm}{0.5cm} 
%\textit{How could one make an adaptable smart home that can handle various devices from different suppliers, combined in an accessible user control interface, that is accessible from a single device %connected to the local Wi-Fi, without the need of a central hub?}
%\end{changemargin} 

%The problem tries to encapsulate the problems found in the three use cases: 
%\begin{itemize}
%\item John's usability problem with using multiple applications to control his home.
%\item Arthur's problem with his defect hub that his entire Smart Home relies on.
%\item Sarah's problem with unstable Internet and dependency on Internet connection, rather than just Wi-Fi connection to the devices. 
%\end{itemize}

%As found in the existing solutions, a solution which is both effective, universal and does not use a \hub, does not exist. Furthermore, only a few solutions have implemented authentication, where none of these implements access levels in their solutions.